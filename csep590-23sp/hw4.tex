\documentclass{article}
\usepackage{algorithm}
\usepackage{algpseudocodex}
\usepackage{graphicx}
\usepackage{amsmath}
\usepackage{xcolor}
\usepackage{listings}
\usepackage{hyperref}

\definecolor{codegreen}{rgb}{0,0.6,0}
\definecolor{codegray}{rgb}{0.5,0.5,0.5}
\definecolor{codepurple}{rgb}{0.58,0,0.82}
\definecolor{backcolour}{rgb}{0.95,0.95,0.92}

\lstdefinestyle{mystyle}{
    backgroundcolor=\color{backcolour},   
    commentstyle=\color{codegreen},
    keywordstyle=\color{magenta},
    numberstyle=\tiny\color{codegray},
    stringstyle=\color{codepurple},
    basicstyle=\ttfamily\footnotesize,
    breakatwhitespace=false,         
    breaklines=true,                 
    captionpos=b,                    
    keepspaces=true,                 
    numbers=left,                    
    numbersep=5pt,                  
    showspaces=false,                
    showstringspaces=false,
    showtabs=false,                  
    tabsize=2
}

\lstset{style=mystyle}

\title{CSEP590 : Applied Cryptography: Homework 4}
\author{Karuna Sagar Krishna}

\begin{document}
    \maketitle

    \section*{Task 1a}
    \begin{verbatim}
        { 1, 2, 3, 4, 6, 8, 9, 11, 12, 13, 16, 17, 
         18, 19, 22, 23, 24, 26, 27, 29, 31, 32, 33, 34 }
    \end{verbatim}

    As seen in class, $Z_{35}^*$ is defined as $\{ a \in Z_{35} : gcd(a, 35) = 1 \}$. So we need to remove numbers that are not coprime with 35. This can also be done by removing multiples of 5 or 7 since $35 = 5*7$. Further, Euler's function $\phi(35) = (5-1)(7-1) = 4*6 = 24$ provides the order of $Z_{35}^*$.

    \section*{Task 1b}
    Since $37$ is odd prime, we know that $Z_{37}^*$ is cyclic. This follows from the lemma discussed in class - group $Z_n^*$ is cyclic iff $n = 2, 4, p^k, 2p^k$ where $p$ is an odd prime.

    2 and 5 are generators of $Z_{37}^*$. I used the following code to check if $a \in Z_{37}^*$ is a generator. Note, the below code only works for prime $p$.

    \begin{lstlisting}[language=Python]
def isgen(x, p):
    l = [1]
    a = 1
    for i in range(0, p-2):
        a = (a * x) % p
        l.append(a)
    print(l)
    l.sort()
    print(l)
    for i in range(0, p-1):
        if l[i] != i+1:
            return False
    return True
    \end{lstlisting}

    \section*{Task 1c}
    Given a cyclic group $G = \langle g \rangle$ of order $n$, we discussed Euler's theorem in class which states that $g^n = e$ where $e$ is the identity element in $G$. We can write $G = \{ g^0, g^1 \dots g^{n-1} \}$. Lets multiply each element of the group with $g$, then we'll have $G' = \{ g.g^0, g.g^1 \dots g.g^{n-1} \} = \{g^1, g^2 \dots g^{n-1}, g^n\}$. Notice, that elements are shifted by one. So compared to $G$, we are missing $g^0 = e$ and we have a new element $g^n$. Since $G$ is a group lets say $g^i \in G$ is inverse of $g$. Suppose $i \ne n-1$ i.e. $0 \le i < n-1$, then $e = g.g^i = g^{i+1}$, but this cannot be true because in $G$, there is a unique identity element $g^0$ and $i+1 \ne 0$. So the only choice is $i = n-1$, then $e = g.g^i = g.g^{n-1} = g^n$. This makes sense because $g^n$ is not part of $G$ and we lost $g^0$ in $G'$. So $g^n = e$ implies that $G = G'$. Hence $g^n = e$.

    Next, using the provided hint, given a cyclic group $G = \langle g \rangle$ of order $n$, we want to prove that $g^y \in G$ where $y \in Z_n$ is a generator if and only if $y$ is coprime with $n$ i.e. $gcd(y, n) = 1$. \textbf{Case 1: $g^y$ is a generator} - lets assume that $y$ is not coprime with $n$ i.e. there exists $d > 1$ such that $y = dy'$ and $n = dn'$. We know $g^n = e$, so $g^{dn'} = e$. Raising both sides by $y'$, we get $g^{dn'y'} = e^{y'}$ which can be simplified to $(g^y)^{n'} = e$. But this is not possible because $g^y$ is a generator, and from above proof, we cannot have $(g^y)^{n'} = e$ for $n' < n$. Hence, $y$ is coprime with $n$. \textbf{Case 2: $y$ is coprime with $n$} - then we have $ya + nb = 1$ from the lemma introduced in class for some $a, b \in Z_n$. So $g^1 = g^{ya + nb} = g^{ya}g^{nb} = (g^y)^a(g^n)^b = (g^y)^ae^b = (g^y)^a$ i.e. we have $g = (g^y)^a$. For any $i \in Z_n$, we can write $g^i = (g^y)^{ai}$. So $g^i \in \langle g^y \rangle$. In other words, $G = \langle g \rangle = \langle g^y \rangle$. Hence $g^y$ is a generator of G.

    Any generator $g' \in G = \langle g \rangle$. So $g' = g^i$ for some $i \in Z_n$. And from above we saw that $g^i$ is a generator iff $gcd(i, n) = 1$. By definition of Euler's function $\phi(n)$ is the number of elements that are coprime with $n$ in $Z_n$. Hence $\phi(n)$ is the number of generators in $G$.

    A group with prime order implies that $|G| = n$ is prime. So from above, we know there are $\phi(n)$ generators. Since $n$ is prime, all numbers in $Z_n$ except 0 are coprime with $n$ i.e. $\phi(n) = n-1$. Hence there are $n-1$ generators in $G$. So any element $g \in G$ except $e$ is a generator of $G$.

    \section*{Task 1d}
    Given $n = pq$ for some primes $p$ and $q$ unknown to us and $\phi(n)$, the question how to we find out the factors $p$ and $q$. In class, we saw that $\phi(n) = \phi(pq) = (p-1)(q-1) = pq - p - q + 1$. This can be rewritten as $p + q = n - \phi(n) + 1$. If we can somehow find $p - q$ then we can easily solve for $p$ and $q$. Consider $(p + q)^2 = p^2 + q^2 + 2pq$, so we can compute $p^2 + q^2 = (p+q)^2 - 2n$. Now, $(p - q)^2 = p^2 + q^2 - 2pq = (p + q)^2 - 2n - 2n = (p + q)^2 - 4n$. So $p - q = \pm \sqrt{(p + q)^2 - 4n}$. Now we know $p + q$ and $p - q$ which can solved to get $p$ and $q$. Hence, we can find the factors in polynomial time.

    \section*{Task 2a}
    $10$ is $46^{-1} \mod 51$ i.e. $46 * 10 = 460 = 1 \mod 51$.

    As we saw in class, the multiplicative inverse exists iff $gcd(46, 51) = 1$. Assuming this is true, we can then compute the inverse using the extended Extended Euclidean algorithm since $51x + 46y = gcd(51, 46)$

    \begin{align*}
        \text{iteration 1} \\
            & a = 51, b = 46 \\
            & a \mod b = 5 \\
        \text{iteration 2} \\
            & a = 46, b = 5 \\
            & a \mod b = 1 \\
        \text{iteration 3} \\
            & a = 5, b = 1 \\
            & a \mod b = 0 \\
            & \text{return} (x = 0, y = 1) \\
        \text{back to iteration 2} \\
            & \text{return} (x = 1, y = 0 - 1 \lfloor 46/5 \rfloor) \\
            & \text{return} (x = 1, y = -9) \\
        \text{back to iteration 1} \\
            & \text{return} (x = -9, y = 1 + 9 \lfloor 51/46 \rfloor) \\
            & \text{return} (x = -9, y = 10) \\
    \end{align*}

    So $y = 10$ is the inverse of $46 \mod 51$. Further, note $gcd(51, 46) = gcd(46, 5) = gcd(5, 1) = gcd(1, 0) = 1$

    \section*{Task 2b}
    $x = 31$ is the solution to $46x = 49 \mod 51$. Since we saw above that $gcd(51, 46) = 1$, we know that there exists an inverse of $46 \mod 51$. And by definition of inverse $aa^{-1} = a^{-1}a = 1$

    \begin{align*}
        46x             & = 49 \mod 51 \\
        46^{-1} 46x   & = 46^{-1} 49 \mod 51 \\
        x               & = 46^{-1} 49 \mod 51 \\
        x               & = 10 * 49 \mod 51 \\
        x               & = 490 \mod 51 \\
        x               & = 31
    \end{align*}

    We can verify this by plugging $x = 31$ in the given equation to see if it satisfied. $46*31 = 1426 = 49 \mod 51$

    \section*{Task 2c}
    \begin{align*}
        gcd(45, 51) & = gcd(51, 45 \mod 51) \\
                    & = gcd(51, 45) \\
                    & = gcd(45, 51 \mod 45) \\
                    & = gcd(45, 6) \\
                    & = gcd(6, 45 \mod 6) \\
                    & = gcd(6, 3) \\
                    & = gcd(3, 6 \mod 3) \\
                    & = gcd(3, 0) \\
                    & = 3
    \end{align*}

    Since, $gcd(45, 51) = 3 \ne 1$, the inverse of $45 \mod 51$ does not exist as discussed in class.

    \section*{Task 2d}
    $45*2 = 90 = 39 \mod 51$, so indeed $x = 2$ is a solution for $45x = 39 \mod 51$. Since $45^{-1}$ does not exist as we saw above, we cannot use the technique we used to solve Task 2b above. This is not a contradiction with Task 2c; it only means we need to find a different way to solve the equation if a solution exists. One alternative is to brute force search $x \in Z_{51}$

    \section*{Task 3}
    As discussed in class, the RSA encryption scheme publishes public key $(n, e)$ and the encryption algorithm is $c = x^e \mod n$ where $x$ is the plain text. The new proposal constructs the plain text as $x = m.r$ where $m \in Z_n$ is the original message to communicate and $r \in Z_n$ is random. The decryption algorithm uses the secret key $(n, d)$ to retrieve the plain text as $c^d \mod n = x$. However, note plain text is $x = m.r$ and $r$ is not shared with the receiver. So it not possible for the receiver to retrieve $m$ from $x$. Hence, the new proposal is not correct.

    Further, since $r$ is chosen from $Z_n$ and $m \in Z_n$, we know that $Z_n$ is not necessarily a group under multiplication. So there may not be an inverse of $r$ and hence it may not be possible to retrieve $m$ even if $r$ somehow shared with the receiver.

    \section*{Task 4a}
    RSA encryption security is dependent on the hardness of finding the factors of $n$. Though $n = pq$ for some primes $p$ and $q$, these primes are not known and finding them is hard. The proposal generates $p = (r - i)$ and $q = (r + j)$ starting from a single random number $r$. As the question describes primes are dense and hence $p$ and $q$ are close to $r$. An attacker can roughly guess $r$ as $r' = \sqrt{pq} = \sqrt{n}$. Then attacker tries to guess $p$ by computing $r' - 1, r' - 2, \dots$ until a prime $p'$ is found and similarly guess $q$ by computing $r' + 1, r' + 2, \dots$ until a prime $q'$ is found. If $p'q' = n$ then we found the factors. If not, we consider other candidates for $p'$ and $q'$. Since $p$ and $q$ are close to $r$, primes are dense and checking for prime can be done efficiently, we can find the factors in reasonable amount of time. Hence, this leads a insecure encryption scheme.

    \section*{Task 4b}
    Found the factors using the following python code which encodes the idea described above.

    \begin{verbatim}
        Found factors 
        p: 35123014591230139123011933120312
            223198716238123918231119382061,
        q: 35123014591230139123011933120312
            223198716238123918231119382447
    \end{verbatim}

    \begin{lstlisting}[language=Python]
        from decimal import *
        from sympy import *
        
        def factor(n):
            with localcontext() as ctx:
                ctx.prec = 124/2
                r = int(Decimal(n).sqrt())
        
                pl = []
                for i in range (1, 1000):
                    p = r - i
                    if isprime(p):
                        pl.append(p)
        
                ql = []
                for i in range (1, 1000):
                    q = r + i
                    if isprime(q):
                        ql.append(q)
                
                for p in pl:
                    for q in ql:
                        if p * q == n:
                            print("Found factors p: {}, q: {}".format(p, q))
                            return
        
        factor(
            12336261539757652568320691057196254494530050076556470009232333
            67120767290238588667397052161653352801437540471197470570083267)
    \end{lstlisting}

\end{document}
